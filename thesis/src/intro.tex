\chapter{Introduzione}
\label{ch:intro}

In un mondo dove l'informatica si è espansa fino a rasentare l'onnipresenza, nuove informazioni vengono prodotte così velocemente e in quantità tale da sopraffare persino la potenza di calcolo dei computer. Perciò, è fondamentale conoscere e padroneggiare delle tecniche per affrontare efficientemente i \textit{big data}, moli di dati così grandi da essere intrattabli con la forza bruta. \\

Questo studio è il naturale proseguimento del lavoro svolto come progetto per il corso di \textit{Data Mining and Organization}\footnote{Si tratta di un esame del Corso di Laurea Magistrale in inforamtica, curriculum \textit{Data Science}, che l'autore di questo documento ha deciso di intraprendere per conseguire i crediti a libera scelta.}. L'attività svolta in quella sede consisteva in una \textit{cluster analysis} dei dati relativi alla carriera universitaria degli studenti del Corso di Laurea Triennale in Informatica. Facendo tesoro di quanto realizzato in quell'occasione, ed ispirandosi in parte a quanto già realizzato in \cite{articolo}, è stata portata avanti un'analisi molto più ampia e approfondita sullo stesso ambito, integrando il materiale iniziale e spostando il focus dagli studenti agli insegnamenti. \\

Questo documento riporta la descrizione di ogni fase delle attività di \textit{data understanding} e \textit{data mining} svolte durante lo studio dei dati a disposizione:

\begin{itemize}
    \item nel Capitolo \ref{ch:rawd}, verrà analizzata e descritta la natura dei dati iniziali;
    \item nel Capitolo \ref{ch:tech}, si illustreranno le tecnologie che compongono la \textit{technology stack} impiegata, con particolare enfasi sulla scelta di esse in relazione al risultato da raggiungere;
    \item nel Capitolo \ref{ch:undst}, si mostrerà il risultato di una fase preliminare di \textit{data understanding} effettuata direttamente sui dati grezzi, allo scopo di intuire quali informazioni possono esserne estratte;
    \item nel Capitolo \ref{ch:prepr}, si descriverà il procedimento di \textit{preprocessing} atto a mutare la forma dei dati iniziali per rendere utilizzabili su di essi alcune tecniche di \textit{data mining};
    \item nel Capitolo \ref{ch:visual}, verranno mostrati e interpretati dei grafici ottenuti mediante alcune tecniche di visualizzazione, atti a evidenziare le tendenze degli attributi generali di ogni classe di record;
    \item nel Capitolo \ref{ch:cluster}, sarà illustrata la fase di \textit{cluster analysis}, volta a cercare delle zone di agglomerazione nello spazio definito dagli attributi dei dati;
    \item nel Capitolo \ref{ch:ass}, si descriverà il procedimento di ricerca di regole associative fra la produttività degli studenti e le valutazioni dei corsi;
    \item nel Capitolo \ref{ch:seq}, verrà mostrato l'utilizzo dell'algoritmo \textit{Generalized Sequential Pattern} per la ricerca di pattern sequenziali frequenti nell'ordine di superamento degli esami;
    \item infine, nel Capitolo \ref{ch:fine}, si cercherà di riassumere quanto rilevato in tutte le analisi e si discuterà di eventuali nuovi possibili sviluppi di questo lavoro.
\end{itemize}

Completa l'opera l'Appendice \ref{appendix}, dove sono riportati il codice e i file significativi per qualche aspetto, ma che non hanno trovato spazio nei relativi capitoli.
