\chapter{Introduzione}
\label{ch:intro}

Questo studio è il naturale proseguimento del lavoro svolto come progetto di \textit{Data Mining and Organization} --- esame del Corso di Laurea Magistrale in Inforamtica, curriculum \textit{data science}, che l'autore di questo documento ha deciso di intraprendere per conseguire i crediti a libera scelta. \\

L'attività svolta in quella sede consiste in una \textit{cluster analysis} dei dati relativi alla carriera universitaria degli studenti del Corso di Laurea Triennale in Informatica. Facendo tesoro di quanto realizzato in quell'occasione, è stata portata avanti una analisi molto più ampia e approfondita sullo stesso ambito, integrando il materiale iniziale e spostando il focus sui corsi di esame. \\

Questo documento riporta una descrizione dettagliata di ogni fase delle attività di \textit{data mining} svolte durante lo studio dei dati a disposizione:

\begin{itemize}
    \item nel capitolo \ref{ch:rawd}, verrà analizzata e descritta la natura dei dati iniziali;
    \item nel capitolo \ref{ch:tech}, si illustreranno le tecnologie che compongono la \textit{technology stack} impiegata, con particolare enfasi sulla scelta di esse in relazione allo scopo da raggiungere;
    \item nel capitolo \ref{ch:undst}, si mostrerà il risultato di una fase preliminare di \textit{data understanding} effettuata direttamente sui dati grezzi, allo scopo di intuire quali informazioni possono essere estratte da essi;
    \item nel capitolo \ref{ch:prepr}, si descriverà il procedimento di \textit{preprocessing} atto a trasformare i dati iniziali per rendere utilizzabili le tecniche di \textit{data mining};
    \item  eh
\end{itemize}

