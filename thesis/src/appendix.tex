\begin{appendices}
\label{appendix}
\chapter{MakeFile e Codice Python}

    \section{Pulizia delle Collection}
    \label{appendix:cleanings}

        Modulo Python per la pulizia delle \textit{collection}: \\
        \lstinputlisting[language=python]{../prepr/mymodules/cleanings.py}

        \vspace{0.8 cm}

        Script che utilizzano il modulo, riferiti nel \texttt{makefile}: \\
        \lstinputlisting[language=python, caption={dataset\_clean.py}]{../prepr/dataset_clean.py}
        \lstinputlisting[language=python, caption={teva\_clean.py}]{../prepr/teval_clean.py}

    \section{Aggregazione degli Attributi}
    \label{appendix:aggregs}

        Modulo Python per l'aggregazione degli attributi: \\
        \lstinputlisting[language=python]{../prepr/mymodules/aggregs.py}

        \vspace{0.8 cm}

        Script che utilizza il modulo, riferito nel \texttt{makefile}: \\
        \lstinputlisting[language=python, caption={teval\_aggr.py}]{../prepr/teval_aggr.py}
        \lstinputlisting[language=python, caption={stud\_aggr.py}]{../prepr/stud_aggr.py}

    \section{Discretizzazione}
    \label{appendix:discretize}

        Modulo Python per la discretizzazione degli attributi: \\
        \lstinputlisting[language=python]{../prepr/mymodules/discretize.py}

        \vspace{0.8 cm}

        Script che contiene le dichiarazioni dei \textit{range} di discretizzazione:
        \lstinputlisting[language=python, caption={PREPR\_PARAMS.py}]{../prepr/PREPR_PARAMS.py}

        \vspace{0.8 cm}

        Script che utilizza il modulo, riferito nel \texttt{makefile}: \\
        \lstinputlisting[language=python, caption={dataset\_discretize.py}]{../prepr/dataset_discretize.py}

    \section{Join}
    \label{appendix:merge}
        Modulo Python per effettuare \textit{join} e \textit{merge} fra due \textit{collection}: \\
        \lstinputlisting[language=python]{../prepr/mymodules/merge.py}

        \vspace{0.8 cm}

        Script che utilizzano il modulo, riferiti nel \texttt{makefile}: \\
        \lstinputlisting[language=python, caption={dataset\_merge.py}]{../prepr/dataset_merge.py}
        \lstinputlisting[language=python, caption={teval\_merge.py}]{../prepr/teval_merge.py}

    \section{Produzione Data Set}
   
        \subsection{Minimizzazione degli Attributi sul Data Set Unito}
        \label{appendix:min}

            Script per l'aggregazione totale della valuazione degli insegnamenti sul data set unito, riferito nel \texttt{makefile}: \\
            \lstinputlisting[language=python, caption={dataset\_min.py}]{../prepr/dataset_min.py}

        \subsection{Condensazione delle Valutazioni sui Insegnamenti}
        \label{appendix:eval}

            Script per l'aggregazione della valuazione degli insegnamenti in un data set minimale, riferito nel \texttt{makefile}: \\
            \lstinputlisting[language=python, caption={dataset\_eval\_gen.py}]{../prepr/dataset_eval_gen.py}

        \subsection{Condensazione della Produttività degli Studenti}
        \label{appendix:stud}

            Script per l'aggregazione della produttività degli studenti in un data set minimale, riferito nel \texttt{makefile}: \\
            \lstinputlisting[language=python, caption={dataset\_stud\_gen.py}]{../prepr/dataset_stud_gen.py}

        \subsection{Sequenze Ordinate di Esami Superati}
        \label{appendix:seq}

            Script per estrarre delle sequenze ordinate di esami superati da ogni studente, in un formato direttamente utilizzabile da Weka: \\
            \lstinputlisting[language=python, caption={dataset\_stud\_gen.py}]{../prepr/dataset_stud_seq.py}

    \section{MakeFile}
    \label{appendix:makefile}
	    \lstinputlisting{../makefile.}
        % on windows - miktex, it wants a dot because of reasons

\end{appendices}
